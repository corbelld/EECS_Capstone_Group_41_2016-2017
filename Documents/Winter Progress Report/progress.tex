\documentclass[onecolumn, draftclsnofoot,10pt, compsoc]{IEEEtran}
\usepackage{graphicx}
\usepackage{url}
\usepackage{setspace}
\usepackage{multirow}

\usepackage{geometry}
\geometry{textheight=9.5in, textwidth=7in}

% 1. Fill in these details
\def \CapstoneTeamName{		Capstone 41}
\def \CapstoneTeamNumber{		41}
\def \GroupMemberOne{			David Corbelli}
\def \GroupMemberTwo{			Jason Ye}
\def \GroupMemberThree{			Zixuan Feng}
\def \CapstoneProjectName{		Website and Application for Health Careers in Oregon}
\def \CapstoneSponsorCompany{	Oregon Department of Education}
\def \CapstoneSponsorPerson{		Art Witkowski}

% 2. Uncomment the appropriate line below so that the document type works
\def \DocType{	%Problem Statement
				%Requirements Document
				%Technology Review
				%Design Document
				Progress Report
				}
			
\newcommand{\NameSigPair}[1]{\par
\makebox[2.75in][r]{#1} \hfil 	\makebox[3.25in]{\makebox[2.25in]{\hrulefill} \hfill		\makebox[.75in]{\hrulefill}}
\par\vspace{-12pt} \textit{\tiny\noindent
\makebox[2.75in]{} \hfil		\makebox[3.25in]{\makebox[2.25in][r]{Signature} \hfill	\makebox[.75in][r]{Date}}}}
% 3. If the document is not to be signed, uncomment the RENEWcommand below
%\renewcommand{\NameSigPair}[1]{#1}

%%%%%%%%%%%%%%%%%%%%%%%%%%%%%%%%%%%%%%%
\begin{document}
\begin{titlepage}
    \pagenumbering{gobble}
    \begin{singlespace}
    	\includegraphics[height=4cm]{coe_v_spot1}
        \hfill 
        % 4. If you have a logo, use this includegraphics command to put it on the coversheet.
        %\includegraphics[height=4cm]{CompanyLogo}   
        \par\vspace{.2in}
        \centering
        \scshape{
            \huge CS Capstone \DocType \par
            {\large\today}\par
            \vspace{.5in}
            \textbf{\Huge\CapstoneProjectName}\par
            \vfill
            {\large Prepared for}\par
            \Huge \CapstoneSponsorCompany\par
            \vspace{5pt}
            {\Large\NameSigPair{\CapstoneSponsorPerson}\par}
            {\large Prepared by }\par
            Group\CapstoneTeamNumber\par
            % 5. comment out the line below this one if you do not wish to name your team
            \CapstoneTeamName\par 
            \vspace{5pt}
            {\Large
                \NameSigPair{\GroupMemberOne}\par
                \NameSigPair{\GroupMemberTwo}\par
                \NameSigPair{\GroupMemberThree}\par
            }
            \vspace{20pt}
        }
        \begin{abstract}
        % 6. Fill in your abstract    
The application and website for Oregon Health Science Careers is a guide for middle school and high school level students to learn about health science careers.  
Between an application and website, the service shall provide information as exploration tool for students looking for future careers in health science. 
Due to the many fields under the classification of health and science, it may be difficult for a student beginning to take an interest in the subject to find the proper path they’re looking for. 
As a development team we will focus on developing an exploratory website and then develop corresponding applications for two major platforms, Android and iOS.
Through our platform, students will explore pathways and careers in health and science. 
In this paper, we will discuss our design and specifications of the applications and website we plan to develop. 
        \end{abstract}     
    \end{singlespace}
\end{titlepage}
\newpage
\pagenumbering{arabic}
\tableofcontents
% 7. uncomment this (if applicable). Consider adding a page break.
%\listoffigures
%\listoftables
\clearpage

% 8. now you write!
\section{Term Recap}

\noindent We began our term with the assignment to create a Website and application for the Oregon Department of Education; this task was given to us by our client, Art Witkowski.
\\ \\
\noindent The following week included a video conference, during which the basic functionality of the site was discussed. Once we had the purpose of the site in mind, we were able to draft a formal document containing our problem statement. This document identifies the underlying problem and how our project intends to offer a solution. 
\\ \\
\noindent Our next step was the drafting of our software requirements specification. This document contains information detailing both deliverable requirements of the final product as well as the needs of our team in project development.
\\ \\
\noindent Once we were able to gain perspective on our site's development requirements, we were able to identify existing technologies useful in our development. These technologies come in the forms of both software and development techniques.
\\ \\
\noindent Before beginning development, it was necessary that we plan our implementation of the desktop site, mobile site, and mobile application required by our project specifications. We were able to identify individual design components and thoroughly document our implementation strategies.
\\ \\
\noindent Towards the end of our term, our assignments consisted of compiling all of our completed documents into our GitHub repository. With our project's design in place we are ready to begin planning our website's page and navigation design.


\section{Retrospective}
\vspace{0.6cm}



\center\textbf{Retrospective Chart}
\vspace{0.3cm}

\begin{tabular}
{| p{0.3\linewidth}| p{0.3\linewidth} | p{0.3\linewidth} | p{0.3\linewidth} |}


\hline Positive Column&Deltas Column&Actions Column\\


\hline 
Productive Group Work Sessions 
&
Becoming more familiar with the technologies related to the project
&
Do more research on the technologies
\\

\hline 
Assignments were able to keep up with due dates 
&
Need to meet with our client more frequently
&
Schedule weekly or biweekly meetings
\\
\hline

\hline 
Assignments were able to keep up with due dates 
&
Need to meet with our client more frequently
&
Schedule weekly or biweekly meetings
\\
\hline
\end{tabular}



\end{document}